\nonstopmode{}
\documentclass[letterpaper]{book}
\usepackage[times,inconsolata,hyper]{Rd}
\usepackage{makeidx}
\usepackage[utf8,latin1]{inputenc}
% \usepackage{graphicx} % @USE GRAPHICX@
\makeindex{}
\begin{document}
\chapter*{}
\begin{center}
{\textbf{\huge Package `davidRian'}}
\par\bigskip{\large \today}
\end{center}
\begin{description}
\raggedright{}
\item[Type]\AsIs{Package}
\item[Title]\AsIs{What the package does (short line)}
\item[Version]\AsIs{0.1.0}
\item[Date]\AsIs{2017-12-12}
\item[Author]\AsIs{Oguzhan Ogreden}
\item[Maintainer]\AsIs{Oguzhan Ogreden }\email{oguzhan@oguzhanogreden.com}\AsIs{}
\item[Description]\AsIs{Utility functions for working with univariate Davidian Curves.}
\item[License]\AsIs{GPL}
\item[Imports]\AsIs{Rcpp (>= 0.12.14)}
\item[LinkingTo]\AsIs{Rcpp, RcppArmadillo}
\item[RoxygenNote]\AsIs{6.0.1}
\end{description}
\Rdcontents{\R{} topics documented:}
\inputencoding{utf8}
\HeaderA{dcGrad}{Gradient of the log likelihood of a univariate DC}{dcGrad}
%
\begin{Description}\relax
Gradient of the loglikelihood of a univariate DC, to be used in estimation.
\end{Description}
%
\begin{Usage}
\begin{verbatim}
dcGrad(x, phi)
\end{verbatim}
\end{Usage}
%
\begin{Arguments}
\begin{ldescription}
\item[\code{x}] A vector of observations.

\item[\code{phi}] DC parameters as introduced in Woods \& Lin.
\end{ldescription}
\end{Arguments}
\inputencoding{utf8}
\HeaderA{ddc}{Density function for univariate Davidian Curves (DC)}{ddc}
%
\begin{Description}\relax
Returns the density for a vector of x.
\end{Description}
%
\begin{Usage}
\begin{verbatim}
ddc(x, mean, sd, phi)
\end{verbatim}
\end{Usage}
%
\begin{Arguments}
\begin{ldescription}
\item[\code{x}] A vector of observations.

\item[\code{mean}] Mean of the normal base of DC, see the package vignette.

\item[\code{sd}] SD of the normal base of DC, see the package vignette.

\item[\code{phi}] DC parameters as introduced in Woods \& Lin.
\end{ldescription}
\end{Arguments}
\inputencoding{utf8}
\HeaderA{rdc}{Random samples from a univariate Davidian Curve (DC)}{rdc}
%
\begin{Description}\relax
Returns n samples from a univariate DC.
\end{Description}
%
\begin{Usage}
\begin{verbatim}
rdc(n, mean, sd, phi)
\end{verbatim}
\end{Usage}
%
\begin{Arguments}
\begin{ldescription}
\item[\code{n}] Number of observations to be sampled.

\item[\code{mean}] Mean of the normal base of DC, see the package vignette.

\item[\code{sd}] SD of the normal base of DC, see the package vignette.

\item[\code{phi}] DC parameters as introduced in Woods \& Lin.
\end{ldescription}
\end{Arguments}
\printindex{}
\end{document}
